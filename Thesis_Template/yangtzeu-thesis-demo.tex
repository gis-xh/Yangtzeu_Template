%% An Introduction to LaTeX Thesis Template of Yangtze University
%%
%% Created by gis-xh

% 导言区

% 使用ctexart类,12pt字号,a4纸张,单页模式
\documentclass[UTF8, 12pt, a4paper, oneside]{ctexart}
% 使用inputenc宏包,设置文档编码方式
% \usepackage[UTF8]{inputenc}
\usepackage{amsmath, amsthm, amssymb, graphicx}
\usepackage[bookmarks=true, colorlinks, citecolor=blue, linkcolor=black]{hyperref}

% 使用geometry宏包,设置页边距
\usepackage{geometry}
\geometry{left=3.0cm,right=3.0cm,top=3.0cm,bottom=2.5cm}
% 设置行间距
\linespread{1.25}
% 设置标准字间距
\usepackage{microtype}

% 设置页眉页脚
\usepackage{fancyhdr} % 引入宏包
\pagestyle{fancy} % 设置页面样式为 fancy
\fancyhf{} % 清除默认的页眉页脚
% 页眉中间显示小五号宋体的章节名称
\fancyhead[C]{\zihao{-5}\songti\leftmark}
% 页脚中间显示小五号宋体的页码
\fancyfoot[C]{\zihao{-5}\songti\thepage}

\usepackage{titlesec} 
% 定义封面内容
% \title{A Sample LaTeX Document} \author{测绘QX221 谢泓} \date{\today}
% % 定义封面样式
% \titleformat{\title}{\centering\Huge\bfseries}{\thetitle}{1em}{}
% \titleformat{\author}{\centering\Large}{\theauthor}{1em}{}
% \titleformat{\date}{\centering}{\thedate}{1em}{}

\begin{document}

%第一个封面,没有图片logo
\begin{titlepage}
    \centering
    \vspace*{5cm} %设置标题距离页面顶部的距离
    \includegraphics[height=1.69cm, width=6.68cm]{./figures/Yangtze_University.png}
    
    \vspace{1cm}
    {\Huge 硕 士 研 究 生 学 位 论 文}\par %设置标题字体大小和换行
    \vspace{1cm} %设置副标题距离标题的距离
    {\Large 基于 Latex 的模板}\par %设置副标题字体大小和换行
    \vfill %填充垂直空白
    {\Large 作者:我}\par %设置作者信息字体大小和换行
    \vspace{0.5cm} %设置日期距离作者信息的距离
    {\Large 论文起止日期:2022年1月1日至}\par %设置日期字体大小和换行
\end{titlepage}

%第二个封面,有图片logo
\begin{titlepage}
    \centering
    \vspace*{3cm} %设置图片logo距离页面顶部的距离
    % 插入图片logo
    \includegraphics[height=2.86cm,width=2.86cm]{./figures/Yangtze_University_logo.png}
    \par
    \includegraphics[height=1.69cm, width=6.68cm]{./figures/Yangtze_University.png}
    
    \vspace{1cm} %设置标题距离图片logo的距离
    {\Huge 这是我的标题}\par %设置标题字体大小和换行
    \vspace{1cm} %设置副标题距离标题的距离
    {\Large 这是我的副标题}\par %设置副标题字体大小和换行
    \vfill %填充垂直空白
    {\Large 作者:我}\par %设置作者信息字体大小和换行
    \vspace{0.5cm} %设置日期距离作者信息的距离
    {\Large 日期:2022年1月1日}\par %设置日期字体大小和换行
\end{titlepage}

\input{pages/abstract} % 引入中文摘要
\newpage % 另起一页

\tableofcontents % 生成目录
\newpage % 另起一页

\input{pages/enabstract} % 引入英文摘要
\newpage % 另起一页

\input{pages/chapter1} % 引入第一章
\newpage % 另起一页

% 攻博期间发表的科研成果目录

在这里列出你攻博期间发表的科研成果 % 引入个人成果
\newpage % 另起一页

\input{pages/thanks} % 引入致谢


\end{document}

